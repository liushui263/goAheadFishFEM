% FEM Maxwell Solver with Electric and Magnetic Dipole Sources
% LaTeX file for direct use in documentation or code reference, with plots included

\documentclass[12pt]{article}
\usepackage{amsmath, amssymb, physics, bm}
\usepackage{geometry}
\usepackage{graphicx}
\geometry{margin=1in}
\begin{document}

\section*{FEM Discretization of 3D Maxwell Equations with Dipole Sources}

% 1. Frequency-domain Maxwell equation
\subsection*{1. Frequency-domain Maxwell Equation (with Dipoles)}
\begin{equation}
\nabla \times (\bm{\mu}^{-1} \nabla \times \bm{E}) - \omega^2 \bm{\epsilon} \bm{E} = j \omega \bm{J}_p + \nabla \times (\bm{m} \delta(\bm{r}-\bm{r}_0))
\end{equation}

Electric dipole source:
\begin{equation}
\bm{J}_p = -j \omega \bm{p} \delta(\bm{r}-\bm{r}_0)
\end{equation}

Magnetic dipole source:
\begin{equation}
\bm{J}_m = \nabla \times (\bm{m} \delta(\bm{r}-\bm{r}_0))
\end{equation}

\includegraphics[width=0.8\textwidth]{images/FEM3D_01.png}

% 2. Weak form for FEM
\subsection*{2. Weak Form (FEM)}
Choose test functions $\bm{v} \in H(\mathrm{curl}, \Omega)$:
\begin{equation}
\int_\Omega (\nabla \times \bm{v}) \cdot \bm{\mu}^{-1} (\nabla \times \bm{E}) , d\Omega - \omega^2 \int_\Omega \bm{v} \cdot \bm{\epsilon} \bm{E} , d\Omega = \int_\Omega \bm{v} \cdot (j \omega \bm{J}_p + \bm{J}_m) , d\Omega
\end{equation}

\includegraphics[width=0.8\textwidth]{images/FEM3D_02.png}

% 3. FEM discretization using Nédélec elements
\subsection*{3. FEM Discretization}
Expand the electric field with Nédélec basis functions:
\begin{equation}
\bm{E} \approx \sum_{j=1}^{N} E_j \bm{N}_j
\end{equation}

The resulting sparse linear system:
\begin{equation}
\sum_{j=1}^N \left[ \int_\Omega (\nabla \times \bm{N}_i) \cdot \bm{\mu}^{-1} (\nabla \times \bm{N}*j) , d\Omega - \omega^2 \int*\Omega \bm{N}_i \cdot \bm{\epsilon} \bm{N}_j , d\Omega \right] E_j = - j \omega \bm{N}_i(\bm{r}_0) \cdot \bm{p} + (\nabla \times \bm{N}_i)(\bm{r}_0) \cdot \bm{m}
\end{equation}

\includegraphics[width=0.8\textwidth]{images/FEM3D_03.png}

% 4. Matrix form for programming
\subsection*{4. Sparse Matrix Form (Programming-Friendly)}
\begin{equation}
\bm{K} \bm{E} = \bm{F}, \quad
\begin{aligned}
\bm{K}*{ij} &= \int*\Omega (\nabla \times \bm{N}_i) \cdot \bm{\mu}^{-1} (\nabla \times \bm{N}*j) , d\Omega - \omega^2 \int*\Omega \bm{N}_i \cdot \bm{\epsilon} \bm{N}_j , d\Omega \
\bm{F}_i &= - j \omega \bm{N}_i(\bm{r}_0) \cdot \bm{p} + (\nabla \times \bm{N}_i)(\bm{r}_0) \cdot \bm{m}
\end{aligned}
\end{equation}

% 5. Implementation notes
\subsection*{5. Implementation Notes}
\begin{itemize}
\item Use tetrahedral or hexahedral Nédélec elements for 3D mesh.
\item Material tensors $\bm{\mu}$ and $\bm{\epsilon}$ may vary per element for anisotropy.
\item Apply PEC/PMC or PML boundary conditions by modifying the matrix $\bm{K}$ accordingly.
\item Electric dipole contributions: $\bm{F}_i^p = -j \omega \bm{N}_i(\bm{r}_0) \cdot \bm{p}$
\item Magnetic dipole contributions: $\bm{F}_i^m = (\nabla \times \bm{N}_i)(\bm{r}_0) \cdot \bm{m}$
\item Solve $\bm{K} \bm{E} = \bm{F}$ using GPU or CPU sparse solvers like cuDSS, PETSc, or MUMPS.
\end{itemize}

\end{document}
