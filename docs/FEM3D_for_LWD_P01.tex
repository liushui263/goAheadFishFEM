\documentclass[11pt]{article}
\usepackage{amsmath,amssymb,amsthm}
\usepackage{bm}
\usepackage{geometry}
\geometry{margin=1in}

\title{Adjoint-State Fréchet Derivatives and Gauss--Newton Hessian for Full-Wave Maxwell Inversion}
\date{}
\begin{document}
\maketitle

\section{Forward Problem}

We consider the frequency-domain Maxwell equation:

\begin{equation}
\nabla \times (\mu^{-1} \nabla \times \bm{E})
- i\omega \sigma \bm{E}
- \omega^2 \varepsilon \bm{E}
=
\bm{f}.
\end{equation}

Define operator:

\[
\mathcal{A}(m)\bm{E} = \bm{f},
\quad
m = (\sigma,\varepsilon,\mu).
\]

---

\section{Misfit Functional}

\begin{equation}
J(m)
=
\frac12
\sum_r
\|
\bm{P}_r \bm{E} - \bm{d}_r
\|^2.
\end{equation}

---

\section{First-Order Linearization}

Let

\[
m \to m + \delta m,
\quad
\bm{E} \to \bm{E} + \delta\bm{E}.
\]

Linearizing:

\[
\mathcal{A}(m)\delta\bm{E}
=
-
D_m\mathcal{A}[\delta m]\bm{E}.
\]

Thus,

\begin{equation}
\delta\bm{E}
=
-
\mathcal{A}^{-1}
D_m\mathcal{A}[\delta m]\bm{E}.
\end{equation}

---

\section{Jacobian Operator}

Define the parameter-to-data map:

\[
\bm{F}(m) = \bm{P}\bm{E}(m).
\]

Linearization:

\[
D\bm{F}[\delta m]
=
\bm{P}\delta\bm{E}
=
-
\bm{P}
\mathcal{A}^{-1}
D_m\mathcal{A}[\delta m]\bm{E}.
\]

Define Jacobian operator:

\[
\mathcal{J}
=
-
\bm{P}
\mathcal{A}^{-1}
D_m\mathcal{A}.
\]

---

\section{Gauss--Newton Hessian}

The Gauss--Newton approximation neglects second-order derivatives of $\bm{E}$.

\[
H_{GN}
=
\mathcal{J}^*
\mathcal{J}.
\]

Substitute expression:

\[
H_{GN}
=
(D_m\mathcal{A})^*
(\mathcal{A}^{-1})^*
\bm{P}^*
\bm{P}
\mathcal{A}^{-1}
D_m\mathcal{A}.
\]

If residual is small:

\[
\bm{P}^*\bm{P}
\approx
\text{data weighting}.
\]

---

\section{Explicit Kernel Representation}

We insert the explicit Fréchet derivatives.

---

\subsection{Derivative w.r.t. $\sigma$}

\[
D_\sigma\mathcal{A}[\delta\sigma]\bm{E}
=
i\omega \delta\sigma \bm{E}.
\]

Thus Jacobian action:

\[
\delta d
=
-
i\omega
\bm{P}
\mathcal{A}^{-1}
(\delta\sigma \bm{E}).
\]

Hence Hessian kernel:

\begin{equation}
H_{\sigma\sigma}(x,x')
=
\omega^2
\Re
\left[
\bm{E}(x)
\cdot
\bm{\Lambda}(x')
\right],
\end{equation}

where $\bm{\Lambda}$ solves adjoint problem.

---

\subsection{Derivative w.r.t. $\varepsilon$}

\[
D_\varepsilon\mathcal{A}
=
\omega^2 \delta\varepsilon \bm{E}.
\]

Thus,

\begin{equation}
H_{\varepsilon\varepsilon}
\sim
\omega^4
(\bm{E}\cdot\bm{\Lambda}).
\end{equation}

High-frequency amplification factor:

\[
H_{\varepsilon\varepsilon}
\propto
\omega^4.
\]

---

\subsection{Derivative w.r.t. $\mu$}

\[
D_\mu\mathcal{A}
=
-
\nabla \times
(\mu^{-2}\delta\mu
\nabla \times \bm{E}).
\]

After adjoint manipulation:

\begin{equation}
H_{\mu\mu}
\sim
\mu^{-4}
(\nabla\times\bm{E})
\cdot
(\nabla\times\bm{\Lambda}).
\end{equation}

---

\section{Block Hessian Structure}

Let

\[
m =
\begin{pmatrix}
\sigma \\
\varepsilon \\
\mu
\end{pmatrix}.
\]

Then Gauss--Newton Hessian has block form:

\[
H_{GN}
=
\begin{pmatrix}
H_{\sigma\sigma} & H_{\sigma\varepsilon} & H_{\sigma\mu} \\
H_{\varepsilon\sigma} & H_{\varepsilon\varepsilon} & H_{\varepsilon\mu} \\
H_{\mu\sigma} & H_{\mu\varepsilon} & H_{\mu\mu}
\end{pmatrix}.
\]

Cross terms arise from mixed Jacobians.

Example:

\[
H_{\sigma\varepsilon}
\sim
\omega^3
(\bm{E}\cdot\bm{\Lambda}).
\]

---

\section{Multi-Frequency Joint Inversion}

For frequencies $\{\omega_k\}_{k=1}^K$:

\[
J(m)
=
\sum_{k=1}^K
J^{(k)}(m).
\]

Total Gauss--Newton Hessian:

\[
H_{GN}
=
\sum_{k=1}^K
H_{GN}^{(k)}.
\]

Each block scales as:

\[
H_{\sigma\sigma}^{(k)} \sim \omega_k^2,
\quad
H_{\varepsilon\varepsilon}^{(k)} \sim \omega_k^4,
\quad
H_{\mu\mu}^{(k)} \sim O(1).
\]

Thus multi-frequency stacking improves conditioning:

\[
H_{total}
=
\sum_k
\mathcal{J}_k^*
\mathcal{J}_k.
\]

---

\section{Compact Operator Form}

\[
H_{GN}
=
\sum_k
(D_m\mathcal{A}_k)^*
(\mathcal{A}_k^{-1})^*
W_k
\mathcal{A}_k^{-1}
D_m\mathcal{A}_k.
\]

---

\section{Key Observations}

\begin{itemize}
\item $\varepsilon$ Hessian scales as $\omega^4$.
\item $\sigma$ Hessian scales as $\omega^2$.
\item Multi-frequency inversion yields block-sum structure.
\item Cross-talk between $\sigma$ and $\varepsilon$ grows with frequency.
\item Helmholtz operator causes indefiniteness in Hessian.
\end{itemize}

\end{document}
