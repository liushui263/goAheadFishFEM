\documentclass[11pt]{article}
\usepackage{amsmath,amssymb,amsthm}
\usepackage{bm}
\usepackage{physics}
\usepackage{geometry}
\geometry{margin=1in}

\title{Adjoint-State Fréchet Derivatives for Full-Wave Frequency-Domain Maxwell Equations}
\author{}
\date{}

\begin{document}
\maketitle

\section{Forward Problem}

We consider the frequency-domain Maxwell system under the 
time convention $e^{-i\omega t}$.

\subsection*{Maxwell Equations}

\begin{align}
\nabla \times \bm{E} &= i\omega \bm{B}, \\
\nabla \times \bm{H} &= \bm{J}_s + \sigma \bm{E} - i\omega \varepsilon \bm{E}, \\
\bm{B} &= \mu \bm{H}.
\end{align}

Eliminating $\bm{H}$ and $\bm{B}$ yields the second-order equation for $\bm{E}$:

\begin{equation}
\nabla \times (\mu^{-1} \nabla \times \bm{E})
- i\omega \sigma \bm{E}
- \omega^2 \varepsilon \bm{E}
=
\bm{f},
\end{equation}

where $\bm{f}$ denotes the source term (electric or magnetic dipole).

Define the forward operator:

\begin{equation}
\mathcal{A}(m)\bm{E} = \bm{f},
\end{equation}

with model parameters

\[
m = (\sigma, \varepsilon, \mu).
\]

---

\section{Misfit Functional}

Assume data are electric field measurements:

\begin{equation}
J(m) = 
\frac{1}{2}
\sum_r 
\left\|
\bm{P}_r \bm{E} - \bm{d}_r
\right\|^2.
\end{equation}

---

\section{Fréchet Derivatives of the Forward Operator}

Let

\[
\sigma \to \sigma + \delta\sigma, \quad
\varepsilon \to \varepsilon + \delta\varepsilon, \quad
\mu \to \mu + \delta\mu.
\]

Correspondingly,

\[
\bm{E} \to \bm{E} + \delta\bm{E}.
\]

Linearizing:

\begin{equation}
\mathcal{A}(m+\delta m)(\bm{E}+\delta\bm{E})
=
\bm{f}.
\end{equation}

Subtracting the original equation and neglecting higher-order terms:

\begin{equation}
\mathcal{A}(m)\delta\bm{E}
+
D_m \mathcal{A}[\delta m]\bm{E}
=
0.
\end{equation}

---

\subsection{Derivative w.r.t. Conductivity $\sigma$}

Only the term $-i\omega \sigma \bm{E}$ depends on $\sigma$.

\[
\delta(-i\omega \sigma \bm{E})
=
-i\omega \delta\sigma \bm{E}.
\]

Therefore,

\begin{equation}
D_\sigma \mathcal{A}[\delta\sigma]\bm{E}
=
i\omega \delta\sigma \bm{E}.
\end{equation}

---

\subsection{Derivative w.r.t. Permittivity $\varepsilon$}

\[
\delta(-\omega^2 \varepsilon \bm{E})
=
-\omega^2 \delta\varepsilon \bm{E}.
\]

Thus,

\begin{equation}
D_\varepsilon \mathcal{A}[\delta\varepsilon]\bm{E}
=
\omega^2 \delta\varepsilon \bm{E}.
\end{equation}

---

\subsection{Derivative w.r.t. Permeability $\mu$}

We consider

\[
\nabla \times (\mu^{-1} \nabla \times \bm{E}).
\]

Using

\[
\delta(\mu^{-1}) = -\mu^{-2} \delta\mu,
\]

we obtain

\[
\delta \left(
\nabla \times (\mu^{-1} \nabla \times \bm{E})
\right)
=
\nabla \times
\left(
-\mu^{-2} \delta\mu
\nabla \times \bm{E}
\right).
\]

Hence,

\begin{equation}
D_\mu \mathcal{A}[\delta\mu]\bm{E}
=
-
\nabla \times
\left(
\mu^{-2} \delta\mu
\nabla \times \bm{E}
\right).
\end{equation}

---

\section{Adjoint-State Formulation}

Define the Lagrangian:

\begin{equation}
\mathcal{L}
=
J(m)
+
\Re \left(
\langle \bm{\Lambda},
\mathcal{A}\bm{E}-\bm{f}
\rangle
\right).
\end{equation}

---

\subsection{Adjoint Equation}

Taking variation w.r.t. $\bm{E}$:

\[
\delta \mathcal{L}
=
\langle
\bm{P}^*(\bm{P}\bm{E}-\bm{d}),
\delta\bm{E}
\rangle
+
\langle
\bm{\Lambda},
\mathcal{A}\delta\bm{E}
\rangle.
\]

Using adjoint operator $\mathcal{A}^\dagger$:

\[
\mathcal{A}^\dagger \bm{\Lambda}
=
\bm{P}^*(\bm{P}\bm{E}-\bm{d}).
\]

For Maxwell operator:

\begin{equation}
\nabla \times (\mu^{-1} \nabla \times \bm{\Lambda})
+ i\omega \sigma \bm{\Lambda}
- \omega^2 \varepsilon \bm{\Lambda}
=
\bm{q}.
\end{equation}

---

\section{Gradient Derivation}

\[
\delta J
=
-
\Re
\langle
\bm{\Lambda},
D_m \mathcal{A}[\delta m]\bm{E}
\rangle.
\]

---

\subsection{Gradient w.r.t. $\sigma$}

\[
\delta J
=
-
\Re
\int
\bm{\Lambda}
\cdot
(i\omega \delta\sigma \bm{E})
\, dV.
\]

\[
=
-
\Re
\int
i\omega \delta\sigma
(\bm{\Lambda}\cdot \bm{E})
\, dV.
\]

Thus,

\begin{equation}
\boxed{
\frac{\partial J}{\partial \sigma}
=
-
\Re
\left(
i\omega \bm{\Lambda}\cdot \bm{E}
\right)
}
\end{equation}

---

\subsection{Gradient w.r.t. $\varepsilon$}

\[
\delta J
=
-
\Re
\int
\omega^2 \delta\varepsilon
(\bm{\Lambda}\cdot \bm{E})
\, dV.
\]

\begin{equation}
\boxed{
\frac{\partial J}{\partial \varepsilon}
=
-
\Re
\left(
\omega^2 \bm{\Lambda}\cdot \bm{E}
\right)
}
\end{equation}

---

\subsection{Gradient w.r.t. $\mu$}

\[
\delta J
=
-
\Re
\int
\bm{\Lambda}
\cdot
\left(
-
\nabla \times
(
\mu^{-2}\delta\mu
\nabla \times \bm{E}
)
\right)
\, dV.
\]

Using integration by parts:

\[
\int
\bm{\Lambda}
\cdot
\nabla \times \bm{X}
=
\int
(\nabla \times \bm{\Lambda})
\cdot
\bm{X}.
\]

Therefore,

\[
\delta J
=
-
\Re
\int
\mu^{-2}
\delta\mu
(\nabla \times \bm{E})
\cdot
(\nabla \times \bm{\Lambda})
\, dV.
\]

\begin{equation}
\boxed{
\frac{\partial J}{\partial \mu}
=
\Re
\left(
\mu^{-2}
(\nabla \times \bm{E})
\cdot
(\nabla \times \bm{\Lambda})
\right)
}
\end{equation}

---

\section{Final Explicit Gradient Expressions}

\begin{align}
\nabla_\sigma J &=
-
\Re
\left(
i\omega
\bm{\Lambda}\cdot \bm{E}
\right), \\
\nabla_\varepsilon J &=
-
\Re
\left(
\omega^2
\bm{\Lambda}\cdot \bm{E}
\right), \\
\nabla_\mu J &=
\Re
\left(
\mu^{-2}
(\nabla \times \bm{E})
\cdot
(\nabla \times \bm{\Lambda})
\right).
\end{align}

\end{document}
