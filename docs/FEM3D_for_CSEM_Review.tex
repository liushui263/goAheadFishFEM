\documentclass[11pt]{article}
\usepackage{amsmath, amssymb}
\usepackage{geometry}
\usepackage{hyperref}
\geometry{margin=1in}

\title{Review of ``Three-Dimensional Adaptive Higher-Order Finite Element Simulation for Geo-Electromagnetics --- A Marine CSEM Example''}
\author{}
\date{}

\begin{document}

\maketitle

\section*{1. Objective}

This paper presents a three-dimensional vector finite element framework 
for frequency-domain geo-electromagnetic modeling, focusing on marine 
Controlled-Source Electromagnetics (CSEM).

The method combines:
\begin{itemize}
\item N\'ed\'elec edge elements,
\item Higher-order polynomial approximation,
\item Adaptive mesh refinement (AMR),
\item Primary/secondary field decomposition.
\end{itemize}

\section*{2. Mathematical Model}

Assuming time dependence $e^{-i\omega t}$, the secondary electric field 
satisfies:

\begin{align}
\operatorname{curl}(\mu^{-1} \operatorname{curl} \mathbf{E}_s)
- i\omega (\sigma - i\omega \varepsilon)\mathbf{E}_s
&=
\operatorname{curl}\left([\mu_p^{-1} - \mu^{-1}] 
\operatorname{curl}\mathbf{E}_p \right) \\
&\quad
- i\omega \left([\sigma_p - \sigma] 
- i\omega[\varepsilon_p - \varepsilon]\right)\mathbf{E}_p
\end{align}

Boundary condition:
\[
\mathbf{n} \times \mathbf{E}_s = 0
\]

\section*{3. Finite Element Discretization}

The approximation reads:

\[
\tilde{\mathbf{E}}_s
=
\sum_{j=1}^n e_j \boldsymbol{\phi}_j
\]

The resulting linear system:

\[
A \mathbf{e} = \mathbf{f}
\]

with

\begin{align}
A_{ij}
&=
\int_\Omega
(\mathrm{curl}\,\phi_i)
\cdot
\mu^{-1}
(\mathrm{curl}\,\phi_j)
\, d^3r
\\
&\quad
-
i\omega
\int_\Omega
\phi_i \cdot
(\sigma - i\omega\varepsilon)\phi_j
\, d^3r
\end{align}

\section*{4. Error Estimation}

Magnetic field approximation:

\[
\tilde{\mathbf{H}}_s
=
(i\omega\mu)^{-1}
\mathrm{curl}\tilde{\mathbf{E}}_s
\]

Error indicator:

\[
\eta_K
=
\int_K
(\hat{\mathbf{H}}_s - \tilde{\mathbf{H}}_s)
\cdot
\mu
(\hat{\mathbf{H}}_s - \tilde{\mathbf{H}}_s)
\, d^3r
\]

\section*{5. Numerical Results}

\begin{itemize}
\item Higher-order elements significantly improve convergence.
\item $p=2$ gives optimal accuracy-cost balance.
\item Bathymetry induces strong 3-D effects.
\item FEM outperforms finite volume near sources.
\end{itemize}

\section*{6. Conclusion}

The paper demonstrates that adaptive higher-order N\'ed\'elec finite elements 
are highly effective for 3-D marine CSEM simulations, particularly in 
geometrically complex environments.

\end{document}
